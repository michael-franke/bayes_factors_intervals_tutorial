\documentclass[
  jou,
  longtable,
  nolmodern,
  notxfonts,
  notimes,
  colorlinks=true,linkcolor=blue,citecolor=blue,urlcolor=blue]{apa7}

\usepackage{amsmath}
\usepackage{amssymb}




\RequirePackage{longtable}
\RequirePackage{threeparttablex}

\makeatletter
\renewcommand{\paragraph}{\@startsection{paragraph}{4}{\parindent}%
	{0\baselineskip \@plus 0.2ex \@minus 0.2ex}%
	{-.5em}%
	{\normalfont\normalsize\bfseries\typesectitle}}

\renewcommand{\subparagraph}[1]{\@startsection{subparagraph}{5}{0.5em}%
	{0\baselineskip \@plus 0.2ex \@minus 0.2ex}%
	{-\z@\relax}%
	{\normalfont\normalsize\bfseries\itshape\hspace{\parindent}{#1}\textit{\addperi}}{\relax}}
\makeatother




\usepackage{longtable, booktabs, multirow, multicol, colortbl, hhline, caption, array, float, xpatch}
\usepackage{subcaption}


\renewcommand\thesubfigure{\Alph{subfigure}}
\setcounter{topnumber}{2}
\setcounter{bottomnumber}{2}
\setcounter{totalnumber}{4}
\renewcommand{\topfraction}{0.85}
\renewcommand{\bottomfraction}{0.85}
\renewcommand{\textfraction}{0.15}
\renewcommand{\floatpagefraction}{0.7}

\usepackage{tcolorbox}
\tcbuselibrary{listings,theorems, breakable, skins}
\usepackage{fontawesome5}

\definecolor{quarto-callout-color}{HTML}{909090}
\definecolor{quarto-callout-note-color}{HTML}{0758E5}
\definecolor{quarto-callout-important-color}{HTML}{CC1914}
\definecolor{quarto-callout-warning-color}{HTML}{EB9113}
\definecolor{quarto-callout-tip-color}{HTML}{00A047}
\definecolor{quarto-callout-caution-color}{HTML}{FC5300}
\definecolor{quarto-callout-color-frame}{HTML}{ACACAC}
\definecolor{quarto-callout-note-color-frame}{HTML}{4582EC}
\definecolor{quarto-callout-important-color-frame}{HTML}{D9534F}
\definecolor{quarto-callout-warning-color-frame}{HTML}{F0AD4E}
\definecolor{quarto-callout-tip-color-frame}{HTML}{02B875}
\definecolor{quarto-callout-caution-color-frame}{HTML}{FD7E14}

%\newlength\Oldarrayrulewidth
%\newlength\Oldtabcolsep


\usepackage{hyperref}




\providecommand{\tightlist}{%
  \setlength{\itemsep}{0pt}\setlength{\parskip}{0pt}}
\usepackage{longtable,booktabs,array}
\usepackage{calc} % for calculating minipage widths
% Correct order of tables after \paragraph or \subparagraph
\usepackage{etoolbox}
\makeatletter
\patchcmd\longtable{\par}{\if@noskipsec\mbox{}\fi\par}{}{}
\makeatother
% Allow footnotes in longtable head/foot
\IfFileExists{footnotehyper.sty}{\usepackage{footnotehyper}}{\usepackage{footnote}}
\makesavenoteenv{longtable}

\usepackage{graphicx}
\makeatletter
\def\maxwidth{\ifdim\Gin@nat@width>\linewidth\linewidth\else\Gin@nat@width\fi}
\def\maxheight{\ifdim\Gin@nat@height>\textheight\textheight\else\Gin@nat@height\fi}
\makeatother
% Scale images if necessary, so that they will not overflow the page
% margins by default, and it is still possible to overwrite the defaults
% using explicit options in \includegraphics[width, height, ...]{}
\setkeys{Gin}{width=\maxwidth,height=\maxheight,keepaspectratio}
% Set default figure placement to htbp
\makeatletter
\def\fps@figure{htbp}
\makeatother







\usepackage{newtx}

\defaultfontfeatures{Scale=MatchLowercase}
\defaultfontfeatures[\rmfamily]{Ligatures=TeX,Scale=1}





\title{How to test hypothesis using Bayes Factor in behavioral sciences}


\shorttitle{Test Bayes Factor}


\usepackage{etoolbox}









\authorsnames[{1},{2}]{Timo B. Roettger,Michael Franke}







\authorsaffiliations{
{Department of Linguistics \& Scandinavian Studies, University of
Oslo},{Department of Linguistics, University of Tübingen}}




\leftheader{Roettger and Franke}



\abstract{Recent times have seen a surge of Bayesian inference across
the behavioral sciences. However, the process of testing hypothesis is
often conceptually challenging or computationally costly. This tutorial
provides an accessible, non-technical introduction that covers the most
common scenarios in experimental sciences: Testing the evidence for an
alternative hypothesis using Bayes Factor through the Savage Dickey
approximation. This method is conceptually easy to understand and
computatioanlly cheap. }

\keywords{statistics, Bayes, Bayes Factor, Savage Dickey, hypothesis
testing, ROPE}

\authornote{\par{\addORCIDlink{Timo B. Roettger}{0000-0003-1400-2739}} 
\par{ }
\par{   The authors have no conflict of interest to declare.    }
\par{Correspondence concerning this article should be addressed to Timo
B.
Roettger, Email: \href{mailto:timo.roettger@iln.uio.no}{timo.roettger@iln.uio.no}}
}

\usepackage{pbalance}
% \usepackage{float}
\makeatletter
\let\oldtpt\ThreePartTable
\let\endoldtpt\endThreePartTable
\def\ThreePartTable{\@ifnextchar[\ThreePartTable@i \ThreePartTable@ii}
\def\ThreePartTable@i[#1]{\begin{figure}[!htbp]
\onecolumn
\begin{minipage}{0.485\textwidth}
\oldtpt[#1]
}
\def\ThreePartTable@ii{\begin{figure}[!htbp]
\onecolumn
\begin{minipage}{0.48\textwidth}
\oldtpt
}
\def\endThreePartTable{
\endoldtpt
\end{minipage}
\twocolumn
\end{figure}}
\makeatother


\makeatletter
\let\endoldlt\endlongtable		
\def\endlongtable{
\hline
\endoldlt}
\makeatother

\newenvironment{twocolumntable}% environment name
{% begin code
\begin{table*}[!htbp]%
\onecolumn%
}%
{%
\twocolumn%
\end{table*}%
}% end code

\urlstyle{same}



\makeatletter
\@ifpackageloaded{caption}{}{\usepackage{caption}}
\AtBeginDocument{%
\ifdefined\contentsname
  \renewcommand*\contentsname{Table of contents}
\else
  \newcommand\contentsname{Table of contents}
\fi
\ifdefined\listfigurename
  \renewcommand*\listfigurename{List of Figures}
\else
  \newcommand\listfigurename{List of Figures}
\fi
\ifdefined\listtablename
  \renewcommand*\listtablename{List of Tables}
\else
  \newcommand\listtablename{List of Tables}
\fi
\ifdefined\figurename
  \renewcommand*\figurename{Figure}
\else
  \newcommand\figurename{Figure}
\fi
\ifdefined\tablename
  \renewcommand*\tablename{Table}
\else
  \newcommand\tablename{Table}
\fi
}
\@ifpackageloaded{float}{}{\usepackage{float}}
\floatstyle{ruled}
\@ifundefined{c@chapter}{\newfloat{codelisting}{h}{lop}}{\newfloat{codelisting}{h}{lop}[chapter]}
\floatname{codelisting}{Listing}
\newcommand*\listoflistings{\listof{codelisting}{List of Listings}}
\makeatother
\makeatletter
\makeatother
\makeatletter
\@ifpackageloaded{caption}{}{\usepackage{caption}}
\@ifpackageloaded{subcaption}{}{\usepackage{subcaption}}
\makeatother

% From https://tex.stackexchange.com/a/645996/211326
%%% apa7 doesn't want to add appendix section titles in the toc
%%% let's make it do it
\makeatletter
\xpatchcmd{\appendix}
  {\par}
  {\addcontentsline{toc}{section}{\@currentlabelname}\par}
  {}{}
\makeatother

%% Disable longtable counter
%% https://tex.stackexchange.com/a/248395/211326

\usepackage{etoolbox}

\makeatletter
\patchcmd{\LT@caption}
  {\bgroup}
  {\bgroup\global\LTpatch@captiontrue}
  {}{}
\patchcmd{\longtable}
  {\par}
  {\par\global\LTpatch@captionfalse}
  {}{}
\apptocmd{\endlongtable}
  {\ifLTpatch@caption\else\addtocounter{table}{-1}\fi}
  {}{}
\newif\ifLTpatch@caption
\makeatother

\begin{document}

\maketitle



\setcounter{secnumdepth}{3}

\setlength\LTleft{0pt}




\section{Introduction}\label{introduction}

To date, the most common quantitative approach across the experimental
sciences is to run an experiment with one or more predictors and
statistically test if there is evidence that these predictors affect the
measured variables. Traditionally, this process has been done by form of
null hypothesis significance testing. Over the last decade or so,
however, we have seen more and more statistical approaches within an
alternative inferential framework: Bayesian inference. Testing
hypothesis within the Bayesian framework is often considered either
conceptually challenging, computationally too costly, or both. This
tutorial provides an accessible, non-technical introduction to Bayesian
hypothesis testing that is easy to understand and computationally cheap.

\section{Motivation and intended
audience}\label{motivation-and-intended-audience}

This tutorial provides a very basic introduction to the topic using R (R
Core Team, 2025). We wrote this tutorial with a particular reader in
mind. If you have used R before and if you have a basic understanding of
linear regression, and Bayesian inference, this tutorial is for you. We
will remain mostly conceptual to provide you with a conceptual tool to
approach hypothesis testing within Bayesian inference.

We just want to give you an impression of how a Bayesian regression
analysis looks and feels. The tutorial covers the essential concepts and
explains how to run and interpret the out- put of a Bayesian regression
analysis using the wonderful R package brms written by Paul Buerkner
(2016). If you don't have any experience with regression modeling, you
will probably still be able to follow, but you might also want to
consider doing a crash course. To bring you up to speed, we recommend
the excellent two-part tutorial by Bodo Winter (2013) on mixed effects
regression in a non-Bayesian ---a.k.a. frequentist--- paradigm. In a
sense, this tutorial could be considered part three of the series
started by Winter. We will for example use the same data set. To
actively follow this tutorial, you should have R installed on your com-
puter (https://www.r-project.org). Unless you already have a favorite
editor for tinkering with R scripts, we recommend to try out RStu- dio
(https://www.rstudio.com). You will also need some packages, which you
can import with the following code:

\subsection{Bayesian inference}\label{bayesian-inference}

Evaluating evidence quantitatively

\subsection{Goals}\label{goals}

\subsection{What is Bayes Factor}\label{what-is-bayes-factor}

\subsection{Approximating BF with Savage
Dickey}\label{approximating-bf-with-savage-dickey}

Politeness Data

\subsection{BF for point null}\label{bf-for-point-null}

\subsection{Sensitivity analysis for different
priors}\label{sensitivity-analysis-for-different-priors}

\subsection{BF for a Region of Practical Equivalence
(ROPE)}\label{bf-for-a-region-of-practical-equivalence-rope}

\subsection{How to chose a ROPE?}\label{how-to-chose-a-rope}

\begin{itemize}
\tightlist
\item
  theoretically derived?

  \begin{itemize}
  \tightlist
  \item
    communicatively relevant?
  \end{itemize}
\item
  standardized effect sizes?
\end{itemize}

\subsection{Sensitivity analysis for different priors and
ROPEs}\label{sensitivity-analysis-for-different-priors-and-ropes}

\subsection{Write up}\label{write-up}

Do's Do think think about sensible priors think about sensible ropes
think about the smallest effect sizes of interest instead of testing
point-0

Don'ts Don't fall into the trap of discrete thresholds. Don't hack ropes






\end{document}
